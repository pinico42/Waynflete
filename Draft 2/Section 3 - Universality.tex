\documentclass[Master.tex]{subfiles}

\begin{document}

So far, we have explored how our two methods of computation can perform many common operations, including data manipulation \& storage, on top of looping and conditional branching. However, having an arbitrary set of abilities that mimic computation is not necessarily the same as a machine being able to evaluate any computable statement; they must participate in the abstract concept of computability. For practicality's sake, we have chosen Turing Machines to be the pinnacle of computability; hence, any machine that can be programmed in such a way that it emulates a Turing machine shall be called a computer. In this section we shall discover that the definition does not rely on Turing machines alone; rather, the idea of computability is a robust, almost Platonic reality, that Turing machines, $\lambda$-Calculus and all other computers, physical or theoretical, participate within. This is the idea of Universality.

A Universal machine is a machine which participates in this ideal - it is able to emulate a Turing machine. This essentially means that a Universal machine, given a specific Turing machine starting configuration, should be able to produce every consecutive following state (i.e. it takes a machine's description table as input, and generates an infinite sequence of the following tape configurations \& the corresponding finishing state for that step). Any machine that can achieve this shall be called a computer.

\subsection{Universality in the $\lambda$-Calculus}
We shall prove that $\lambda$-Calculus is Turing complete by building a universal machine using only $\lambda$-functions. The core of the machine will be a function which calculates a single step in the Turing machine's calculation: It will receive as its parameters the current configuration of the tape and the current state. The function will then be applied to itself repeatedly to produce the sequences of states. Due to the complexity of this idea, not every step will be explained in full here; however, a full explanation can be found in appendix \ref{appendix:lambdauniversaldemo}.




\subsection{Conclusion}

We have successfully shown here that $\lambda$-Calculus is a Universal, Turing-complete language - it is able to perform every action that a Turing Machine can perform, and as such is at least as powerful as Turing's method of computation. Furthermore - and this the reader must take on faith, or rely on intuition - it is possible for a Turing machine to simulate $\lambda$-Calculus; Turing machines are Effectively Calculable, and are at least as powerful as the $\lambda$-Calculus. Therefore, these two methods of computation - one based in physical ideas, one based in conceptual ideas, are totally equivalent. Intuitively, it is clear that both use similar logical processes, similar to the workings of the logical aspect of human thought, and as such this result is unsurprising. However, what is truly remarkable is that what appears to be a Platonic reality of the concept of computability has been stumbled upon by two different methods designed independently of each other. This strengthens the idea of computability as defined this way, and gives us a very solid foundation from which to base our exploration of physical machines.

\end{document}