\documentclass {article}
\usepackage[a4paper, total={6in, 8in}]{geometry}
\usepackage[utf8]{inputenc}
\usepackage[english]{babel}
\usepackage{subfiles}
\usepackage{blindtext}
\usepackage{amsmath}
\usepackage{amssymb}
\usepackage{amsthm}
\usepackage{bm}

\title{How Significant was WW2 in the Development of the Computer?}
\date{November 24, 2016}
\author{Max Penrose}


\begin{document}
\pagenumbering{roman}
\maketitle
\tableofcontents

\section{Effective Computability}

In this first section I would like to describe the fundamental attributes a machine must have for it to be called a 'computer'. Prior to the war and the development of electronic computation, any human who performed a calculation would  frequently be called a 'computer'. However, this specific attribute of computation is difficult to define - to do so, I will be looking at the minimalist  $\lambda$-Calculus developed by Mathematician and Logician Alonzo Church, the famous concept of the Universal Turing Machine created by English Mathematician Alan Turing, and the similarities and differences between these two definitions.


\subsection{Church's $\lambda$-Calculus}

\subfile{"Section 1 - Lambda Calculus"}

\end{document}