\documentclass {article}
\usepackage[a4paper, total={6in, 8in}]{geometry}
\usepackage[utf8]{inputenc}
\usepackage[english]{babel}
\usepackage{subfiles}
\usepackage{amsmath}
\usepackage{amssymb}
\usepackage{amsthm}
\usepackage{blindtext}
\usepackage{bm}
\usepackage{cleveref}
\usepackage{hhline}
\usepackage{listings}
\usepackage[bb=boondox]{mathalfa}
\usepackage{multirow, bigdelim}
\usepackage{tabu}
\usepackage{yfonts}


\title{Did WW2 Produce an Equivalent to a Modern Computer?}
\date{February 23, 2017}
\author{Max Penrose}


\crefname{section}{§}{§§}
\Crefname{section}{§}{§§}

\begin{document}
\pagenumbering{roman}
\maketitle
\tableofcontents
\clearpage
\textit{TODO: introduction - stating aim of checking whether computers produced during and after WW2 were 'computers'}

In the process of defining computability, we shall study the set of problems known as `computable functions' - how to describe them, and how to evaluate them. 

\section{Church's $\lambda$-Calculus}

\subfile{"Section 1 - Lambda Calculus"}

\section{Turing Machines}

\subfile{"Section 2 - Turing Machines"}

\section{Universality}

\subfile{"Section 3 - Universality"}
\bibliography{bibliography}
\bibliographystyle{unsrt}
\end{document}