\documentclass[Master.tex]{subfiles}


\begin{document}
\appendix
\appendixpage
\section{Further Arithmetic in $\lambda$-Calculus}
Multiplication can also be calculated by defining the function \textbf{M}:
\cite{rojas2015lambdatutorial}
\begin{equation*}
\bm{\mathrm{M}} = \lambda wyx.w(yx)
\end{equation*}  
\begin{gather*}
\begin{aligned}
\bm{\mathrm{M22}} &= (\lambda wyx.w(yx))(\lambda ab.a(a(b)))(\lambda sz.s(s(z)))\\
&= \lambda x.(\lambda ab.a(a(b)))[(\lambda sz.s(s(z)))x]\\
&= \lambda x.(\lambda ab.a(a(b)))[\lambda z.x(x(z))]\\
&= \lambda x.(\lambda b.[\lambda z.x(x(z))]([\lambda z.x(x(z))]b))\\
&= \lambda x.(\lambda b.[\lambda z.x(x(z))][x(x(b))])\\
&= \lambda x.(\lambda b.x(x(x(x(b)))))\\
&= \lambda xb.x(x(x(x(b))))\\
&= \bm{\mathrm{4}}
\end{aligned}
\end{gather*}
We should rearrange our earlier addition method into an `addition' function of the same format which takes two arguments:
\begin{equation*}
\bm{\mathrm{A}} = \lambda ab.b\bm{\mathrm{S}}a
\end{equation*}

An `exponent' function to find $a^b$can also be defined, of astonishing simplicity:
\cite{penrose1991emperor}
\begin{equation*}
\bm{\mathrm{E}} = \lambda ba.ab
\end{equation*}
\begin{gather*}
\begin{aligned}
\bm{\mathrm{E23}} &= (\lambda ab.ab)(\lambda sz.s(s(z)))(\lambda ty.t(t(t(y))))\\
&= (\lambda sz.s(s(z)))(\lambda ty.t(t(t(y))))\\
&= (\lambda z.[\lambda ty.t(t(t(y)))]([\lambda ux.u(u(u(x)))]z))\\
&= (\lambda z.[\lambda ty.t(t(t(y))][\lambda x.z(z(z(x)))])\\
&= (\lambda z.[\lambda y.[\lambda x.z(z(z(x)))]([\lambda x.z(z(z(x)))]([\lambda x.z(z(z(x)))]y))])\\
&= (\lambda z.[\lambda y.[\lambda x.z(z(z(x))]([\lambda x.z(z(z(x)))][z(z(z(y)))])])\\
&= (\lambda z.[\lambda y.[\lambda x.z(z(z(x))]([z(z(z(z(z(z(y))))))])])\\
&= (\lambda z.[\lambda y.[z(z(z(z(z(z(z(z(z(y)))))))))]])\\
&= \lambda zy.z(z(z(z(z(z(z(z(z(y)))))))))\\
&= \bm{\mathrm{9}} = \bm{\mathrm{3^2}}
\end{aligned}
\end{gather*}
\end{document}