%% It is just an empty TeX file.
%% Write your code here.
\documentclass {article}
\usepackage[a4paper, total={6in, 8in}]{geometry}
\usepackage{hyperref}
\usepackage{xcolor}
\usepackage{amsmath}
\usepackage{bm}

\hypersetup{
	colorlinks=true,
	urlcolor=blue!70!black
}

\title{Bibliography}
\date{2016-11-07}
\author{Max Penrose}

\begin{document}
\pagenumbering{roman}


\section{Effective Computability}

In this first section I would like to describe the fundamental attributes a machine must have for it to be called a 'computer'. Prior to the war and the development of electronic computation, any human who performed a calculation would  frequently be called a 'computer'. However, this specific attribute of computation is difficult to define - to do so, I will be looking at the minimalist  $\lambda$-Calculus developed by Mathematician and Logician Alonzo Church, the famous concept of Universal Computation and  Computable Numbers developed by English Mathematician Alan Turing, and the work of [OTHERS]


\subsection{Church's $\lambda$-Calculus}

The simplest, and in my opinion the most beautiful definition of effective calculability was introduced by the American mathematician Alonzo Church in the 1930s, as a purely mathematical and abstract definition of computability, far from the very real and mechanical world of Turing. It deals with a set of objects - for our purposes denoted with alphanumeric characters - which define what school mathematics or common computer science would call 'functions'. These functions take as their arguments (other objects which the functions act upon) functions alone, and can also output only functions. Take, for example the function $f$:
\begin{equation*}
f = ab
\end{equation*}
This particular function $f$ applies the function $a$ to the function $b$. Note that while commonly we would use the syntax $\mathit{a(b)}$ to express the same thing, it greatly simplifies expression to use this format. Church defines these functions through the method of 'abstraction' - let us look at a different function:
\begin{equation*}
\lambda u.tu
\end{equation*}
This function takes one single argument, which it names the abstract name $u$. It then applies the function $t$ to $u$ and outputs the result. For example, if we were to apply this function to the function $a$:
\begin{equation*}
(\lambda u.tu) a = ta	
\end{equation*}
Functions can take multiple arguments through the use of nesting:
\begin{gather*}
\begin{aligned}
\lambda a.(\lambda b.ba) &=  \lambda ab.ba \\
([\lambda a.(\lambda b.ba)]u)v &= (\lambda b.bu)v \\ &= vu
\end{aligned}
\end{gather*}

Church applies this medium to a variety of concepts we are familiar with. His definitions of numbers are:
\begin{equation*}
\begin{aligned}
\bm{\mathrm{0}} &= \lambda sz.z,
  & % separation
\bm{\mathrm{1}} &= \lambda sz.sz,
  \\
\bm{\mathrm{2}} &= \lambda sz.s(sz),
  &
\bm{\mathrm{3}} &= \lambda sz.s(s(sz)),
  &
etc.
\end{aligned}
\end{equation*}
Here it can bee seen that for each number \textit{n} the argument $s$  is applied to the argument $z$ \textit{n} times.
From this, it is not difficult to define an operation that takes a number as input and increments it by 1 - such a function would take three arguments (the first being the number, and the second two being the arguments $s$ and $z$ for the number) and would apply the argument $s$ to the number:
\begin{equation*}
\bm{\mathrm{S}} = \lambda wyx.y(wyx)
\end{equation*}
Let us test our new successor function by applying it to \textbf{2}:
\begin{gather*}
\begin{aligned}
\bm{\mathrm{S2}} &= (\lambda wyx.y(wyx)) (\lambda sz.s(s(z)))\\
&= y([\lambda sz.(s(s(z)))]yx)\\
&=y[y(y(z))]\\
&=s(s(s(z)))\\
&=\bm{\mathrm{3}}
\end{aligned}
\end{gather*}

From this concept, we are not far from the nursery-school concept of addition. Remembering that each number applies the function $s$ to $z$ \textit{n} times, and that addition is simply incrementing \textit{a} \textit{b} times, addition is as simple as:
\begin{equation*}
a\bm{\mathrm{S}}b	
\end{equation*}
Let us test this by summing \textbf{2} and \textbf{3}:
\begin{gather*}
\begin{aligned}
\bm{\mathrm{2S3}} &= (\lambda sz.s(s(z))) (\lambda wyx.y(wyx)) (\lambda sz.s(s(s(z))))\\
&= (\lambda wyx.y(wyx)) [(\lambda wyx.y(wyx)) [\lambda sz.s(s(s(z)))]]\\
&= (\lambda wyx.y(wyx)) [\lambda sz.s(s(s(s(z))))]\\
&= \lambda sz.s(s(s(s(s(z)))))\\
&= \bm{\mathrm{5}} \qed
\end{aligned}
\end{gather*}



\end{document}

